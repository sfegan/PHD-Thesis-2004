%##############################################################################
%#                                                                            #
%#                       TITLE AND PREAMBULARY SECTIONS                       #
%#                                                                            #
%##############################################################################

\thesistitle{\Mytitle}
\author{Stephen Fegan}
\degree{Doctor of Philosophy}
\department{Department of Physics}
\major{Physics}
\signaturelines{5}     %max number of signature lines is 7
\thadviser{T.C.~Weekes}
\memberone{M.~Shupe}
\membertwo{F.~Melia}
\memberthree{J.D.~Garcia}
\memberfour{K.C.~Hsieh}

\submitdate{2 0 0 4}

% Print titlepage and other prefatory material:
%
\titlepage
\approvalpage
\statementpage

\chapter*{Acknowledgments}
\begin{singlespace}

As everyone who has gone through the process will know, it is
impossible to make such an undertaking without the wisdom and guidance
of an adviser who knows all there is to know about the subject. Thank
you Trevor for all your patience and encouragement.

I would like to particularly emphasize the formative influence on my
first years here of Vladimir Vassiliev and Mike Catanese, two of the
best scientists with whom I have had the pleasure of working. Thanks,
Mike, for sharing your office and for all your patience with my
incessant questioning. Thanks, Vladimir, for involving me in so many
interesting projects and for encouraging me to be rigorous in my
work. I look forward to working with you again soon.

Many thanks to all the visitors to the Whipple observatory, in
particular to John Finley, Stella Bradbury, Stephane LeBohec and Glenn
Sembroski. To all the other students and postdocs with whom I had the
pleasure of working: Jojo, Andrew, Tony, Mead, Martin and Abe. To Ken
who somehow doesn't fit in the last category or the next. To the
staff of FLWO past and present: Steve, Karen, Grace, Ginnie, Danny,
Roger, Dave, Ceasar, Gene, Emmet and Kevin. To Leslie, Dale and Ann at
SAO.

Thanks to all my friends here: Matt, Kelly, Jeff, Taryn, Matty, Janna,
Kevin and to Jack for introducing us. Without you all, life in Tucson
would not have been any fun. I hope that we all remain friends for a
long time to come, and that there are many more trips to Vegas. Thanks
Araby, Jenny, Erin and all at the R.G. for the seemingly endless flow
of beer. Thanks to all my friends in Ireland, especially to Ronan, for
visiting and for putting up with me being out of touch for such long
periods.

To Deirdre for being such a good friend for so many years and for
all the encouragement throughout the writing of this dissertation.

To my parents, Sylvia and David, who have always supported and
encouraged me. It is redundant to say this: without you
both I would never have got this far. To Eoin for all the CDs, robots
and other strange stuff. Looking forward to watching the classics
modern cinema with you again soon, i.e.\ \textit{Top Secret!} and
\textit{Ghostbusters}.

{\scriptsize\begin{singlespace} This research has made use of the
NASA/IPAC Extragalactic Database (NED) which is operated by the Jet
Propulsion Laboratory, California Institute of Technology, under
contract with the National Aeronautics and Space Administration. This
publication makes use of data products from the Two Micron All Sky
Survey, which is a joint project of the University of Massachusetts
and the Infrared Processing and Analysis Center/California Institute
of Technology, funded by the National Aeronautics and Space
Administration and the National Science Foundation. This research has
made use of NASA's Astrophysics Data System.

This research was supported by grants from the U.S. Department of
Energy, PPARC (U.K.) and Enterprise Ireland. The author acknowledges
the support of the Predoctoral Fellowship program at the Smithsonian
Astrophysical Observatory.
\end{singlespace}}


\end{singlespace}

%\chapter*{Dedication}
\newpage
\begin{singlespace}
~

\vspace*{\fill}
\centerline{For my parents.}
\vspace*{\fill}\vspace*{\fill}\vspace*{\fill}
\end{singlespace}

\setcounter{tocdepth}{1}
\tableofcontents
\listoftables          % required if there are tables
\listoffigures         % required if there are figures

\begin{abstract}

A survey of unidentified 100\,MeV \Gray sources is undertaken, with
the Whipple 10\,m telescope, with the objective of detecting very high
energy ($>350$\,GeV) \Gray emission. The survey consists of nineteen
sets of observations of sources detected by the EGRET instrument on
the Compton Gamma-Ray Observatory between 1991 and 1995. Results for
21 EGRET sources are reported; in some cases two EGRET sources are
close enough to be viewed in a single observation. For each EGRET
source, candidate associations are listed and the implications of each
candidate for VHE emission discussed. Finally, a study of the
performance of a next-generation ground based instrument, VERITAS,
using simulations is presented. The implications of the increased
sensitivity of such an instrument for suture \Gray surveys is briefly
discussed.

\end{abstract}
