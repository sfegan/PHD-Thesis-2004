\chapter{Conclusions}
\label{CHAP::CONCLUSIONS}

In the time since the end of the CGRO mission, multiwavelength
observations have proved to be the most powerful tool available to
investigate the origin of the high energy emission from the
unidentified sources. For a number of such sources, some of which were
discussed in chapter~\ref{CHAP::OBSERVATIONS}, x-ray, radio and
optical observations have narrowed the list of scientifically viable,
potential candidates. In some cases, such observations have ruled out
all but one candidate. This survey was undertaken in the hope that VHE
emission would be detected from one of the sources chosen, and that
the higher spatial resolution achievable with the ground-based
technique would allow the source of the \Gray emission to be
identified. There is significant overlap between the VHE source
catalog and the EGRET sources; seven of the eighteen credible VHE
sources were also seen by EGRET at some level. Of the two categories
of sources unambiguously identified by EGRET, blazars and pulsars,
detections of eight BL~Lac type blazars have been claimed at TeV
energies. No pulsars have been directly detected by ground-based
instruments, but some EGRET pulsars are associated with PWN which,
like the Crab, Vela and PSR~1706, may be visible to VHE \Gray
instruments.

In total, results from VHE observations of 21 EGRET sources are
reported, more than 10\% of the unidentified source population. The
observations yielded an average of 5 hours of data from each source.
The decision to obtain this amount of data on a relatively large
number of sources was partly made so that the survey could co-exist
with other observing programs using the Whipple instrument, i.e.\ a
request was not made for a large quantity of data from any one
location in the sky, which would preclude other observation programs
in that area. It was anticipated that this level of observations would
be sufficient to provide upper-limits on emission that would constrain
the spectrum of a mean EGRET source
(figure~\ref{FIG::INTRODUCTION::SENSITIVITY}).

When the survey was initiated, little was known about many of the
observed sources, outside of what was published in the 3EG
catalog. Since this time, our understanding of these sources has
advanced considerably, both through work on the population as a whole
and through multiwavelength observations of individual sources. Of
particular note in the first category is the calculation of source
variability by \citet{REF::NOLAN::APJ2003} and the systematic
correlation of the sources with radio sources
\citep{REF::MATTOX::APJS2001} and SNR, OB-associations and massive
stars \citep{REF::ROMERO::AA1999}. That so many of the sources chosen
here now have potential associations, as discussed in
chapter~\ref{CHAP::OBSERVATIONS}, is a testament to the ongoing
interest in multiwavelength observations of these sources, such as the
work of \citet{REF::ROBERTS::APJS2001},
\citet{REF::MUKHERJEE::APJ2000} and \citet{REF::KAARET::APJ1999}, to
mention just a few.

Based on the number of observations made (i.e.\ the number of sources
surveyed and the number of independent bins in each two-dimensional
image) it cannot be claimed that VHE \Gray emission was detected from
any of the sources, at a significant level. Two of these sources,
3EG~J1337$+$5029 and 3EG~J2227$+$6122, have excesses with sufficiently
low chance probability that they would be considered as suggestive of
\Gray emission, if the observations are taken in isolation from the
rest of the survey. In the case of J1337$+$5029, the excess
corresponds to the location of a cluster, Abell~1758. If excess is the
result of \Gray emission from the cluster, it would represent a new
class of VHE emission and be the most distant source of VHE emission
to date (at $z=0.279$, considerably more distant than H1426$+$428, the
most distant VHE blazar, at $z=0.129$) and have important implications
for the density of the IIRF. The excess in the case of J2227$+$6122
does not correspond to any of the suggested associations for the EGRET
source. To confirm (or refute) any emission, independent follow-up
observations will be made. The excesses correspond to fluxes of 0.40
and 0.33 of the integral Crab Nebula flux, at energies $>350$\,GeV,
respectively. At this level, a five to ten hour exposure on each will
be sufficient for confirmation.

The next generation of ground-based instruments, such as VERITAS, will
be $>10$ times more sensitive than the Whipple 10\,m instrument
(figure~\ref{FIG::INTRODUCTION::SENSITIVITY} and
\ref{FIG::VERITAS::SENSITIVITY}). They will be most sensitive to
\Grays at approximately $\sim$100\,GeV, with some sensitivity even
below this energy (table~\ref{TABLE::VERITAS::DIFFRATE}). A survey of
EGRET sources with one of these instruments, should have considerable
success in detecting \Gray emission.

The EGRET sources J0010$+$7309 and J0634$+$0521, which are associated
with the CTA~1 and Monoceros SNR, are prime candidates for observation
with a next generation instrument. In particular, in conjunction with
the next generation of space-based instruments, VERITAS may resolve
two components of emission from J0634$+$0521, and confirm the model of
\citet{REF::TORRES::PR2003}. Some models suggest that the \Gray source
3EG~J0241$+$6103 (the COS-B source 2CG~135), which likely corresponds
to the x-ray binary system LSI~$+$61$^\circ$303, may be detectable
in the VHE regime.

Of the likely AGN surveyed, 3EG~J0433$+$2908 and GeV~J0508$+$0540,
from which $>10$\,GeV photons were detected by EGRET, are worthy of
follow-up observations with VERITAS, especially as part of a broad
spectrum multiwavelength campaign. Even if not detected in the VHE
regime, the spectra of \Grays from these sources may have implications
for models of the intergalactic infra-red field.

The pulsar candidates surveyed represent another class of objects that
may by detectable with VERITAS, especially if they are associated with
PWN. These sources are probably not good candidates for an initial
round of unidentified EGRET observations; instead observations of
other well known, bright pulsars will hopefully resolve between the
two models of HE pulsar emission. If it is the case that VHE emission
is observed from thes objects, 3EG~J2227$+$6122 and J1826$-$1302 would
represent good candidates for observation.

The population of unidentified sources represent an important enduring
legacy of the EGRET mission, and will remain somewhat of a mystery for
a number of years to come. The next NASA \Gray instrument, GLAST,
scheduled for launch in 2007, will have a point-source flux
sensitivity greater than 50 times that of EGRET
($1.6\times10^{-9}$\,cm$^{-2}$\,s$^{-1}$ at energies $>100$\,MeV in
all-sky survey mode) and a localization accuracy between 0.4 and
5.0\,arcmin. GLAST will undoubtedly identify some fraction of the
unidentified sources but will almost certainly produce a population of
its own unidentified sources close to its flux sensitivity. In the
mean time, the role of multiwavelength observations in studying the
EGRET sources, including more sensitive x-ray, radio, optical and VHE
\Gray observations, cannot be overstated.
