\chapter{Glossary}
\label{APP::JARGON}

\textbf{ACT} --- \textit{Atmospheric \Cerenkov Telescope}, ground-based
\Gray detection technique utilizing the production of \Cerenkov radiation
by charged secondaries (largely e$^\pm$) in the extensive air-showers
that result from interaction of the primary in the atmosphere.

\textbf{AGN} --- \textit{Active Galactic Nucleus}, a galaxy with a powerful
central core which is typically more luminous than the stars of the
host galaxy combined. AGN are sub-categorized by their observational
characteristics, such as the strength of radio emission, variability
and presence or absence of broad emission line. In the unified theory
of AGN, emission is the result of accretion onto a super-massive black
hole, the various classes arising largely through differences in the
orientation with respect to the line of sight of the observer.

\textbf{ASCA} --- Japanese x-ray satellite (1993--2000).

\textbf{Blazar} --- Sub-class of AGN characterized by strong radio 
emission, extreme variability, polarization at radio and optical
wavelengths, and strong continuum emission. Blazars are classified as
either FSRQ or BL~Lac objects, distinguished by the presence (FSRQ) or
absence (BL~Lac) of absorption and emission lines. It is thought that
blazars are AGN with a jet emanating from the core, oriented in the
direction of the observer. They have a two peaked emission spectra,
with correlated synchrotron and inverse-Compton components.

\textbf{BL Lac} --- A type of blazar characterized by the absence of 
absorption and emission lines which makes the determination of
redshift difficult. Their featureless spectra at optical wavelengths
mean that BL~Lacs are usually identified at x-ray or radio energies.
Traditionally BL~Lacs have been classified as low-frequency (LBL) or
high-frequency (HBL) depending on the energy of the peak of
synchrotron emission. There is probably a sequence of intermediate
BL~Lacs which are more difficult to identify as they do not stand out
at radio or x-ray energies. All extragalactic VHE \Gray sources
detected to date are extreme HBLs.

\textbf{CANGAROO} --- \textit{Collaboration between Australia and Nippon 
for a Gamma Ray Observatory in the Outback}, arguably the most
contrived of astronomical acronyms. An ACT experiment operating in the
Australian outback. The group is upgrading their single telescope to an
array of four 10\,m instruments.

\textbf{CGRO} --- \textit{Compton Gamma-Ray Observatory}, second in
NASA's program of ``great observatories''. Launched in 1991 with four
experiments covering the energy range from 60\,keV to 30\,GeV, it
operated for nine years.

\textbf{Chandra} --- Third of NASA's ``great observatories'', an
x-ray instrument named for Subrahmanyan Chandrasekhar (1999-present).

\textbf{COS-B} --- First dedicated European \Gray satellite (1975--1982).
Successful mission, operated in the energy range of 2\,keV to 5\,GeV, 
producing a catalog of sources and detailed observations of Geminga.

\textbf{DSA} --- \textit{Diffusive Shock Acceleration}, acceleration
of a charged particle which repeatedly crosses of a shock-front due to
scattering in the plasma.

\textbf{EGRET} --- \textit{Energetic Gamma-Ray Experiment Telescope},
an instrument on the CGRO satellite, which operated in the energy
range of 30\,MeV to 30\,GeV. The most successful \Gray mission to date,
its many achievements included a catalog of 271 point sources. EGRET
sources are conventionally prefixed by 3EG.

\textbf{erg} --- unit of energy in the CGS system equaling $10^{-7}$\,J.

\textbf{HBL} --- see BL~Lac.

\textbf{HE} --- \textit{High Energy}, in the context of this work, refers 
to the energy range accessible to satellite based \Gray instruments,
30\,MeV to 30\,GeV.

\textbf{HEGRA} --- \textit{High-Energy Gamma Ray Astronomy}, European
ACT and air-shower array experiment on La Palma. The HEGRA group were
the first to successfully employ the stereoscopic technique to
discriminate between \Grays and cosmic-rays.

\textbf{IC} --- \textit{inverse-Compton} scattering.

\textbf{IIRF} --- \textit{Intergalactic Infra-Red Radiation Field}, 
ambient field of infra-red photons that permeates the universe.

\textbf{ISM} --- \textit{Interstellar Medium}, low density material
that permeates the regions between stars in the galaxy.

\textbf{LBL} --- see BL~Lac.

\textbf{MC} --- \textit{Monte Carlo}.

\textbf{O-type star} --- Massive, hot star. Stellar sequence goes
O-B-A-F-G-K-M in order of decreasing surface temperature.

\textbf{OB association} --- Region of the galaxy which has a significant
enhancement in the density of O- and B-type stars. Region has
accelerated rate of star formation and supernovae.

\textbf{PMT} --- \textit{Photo-Multiplier Tube}.

\textbf{PSR} --- prefix used frequently to designate pulsars, e.g.\ 
PSR~1959$+$650, pulsar at sky coordinates $\alpha=19^h59^m$,
$\delta=+65.0^\circ$.

\textbf{PWN} --- \textit{Pulsar Wind Nebula}, synchrotron nebula or
plerion. A supernova remnant which is being resupplied with high
energy electrons by a central pulsar. The electrons cool quickly
through synchrotron emission. For example: The Crab Nebula.

\textbf{RASS-BSC (-FSC)} --- \textit{ROSAT All Sky Survey - Bright 
(Faint) Source Catalog}.

\textbf{ROSAT} --- \textit{R\"ontgen Satellite}, a German-US x-ray
satellite which operated from 1990 to 1999. Its principal instrument,
denoted HRI, operated in the energy range of 0.12\,keV to
2.4\,keV. The main aim mission was the first all-sky survey with a
sensitivity 1000 higher than that of UHURU. ROSAT sources are
conventionally prefixed by RX or 1RXS.

\textbf{RXTE} --- \textit{Rossi X-ray Timing Explorer}, NASA x-ray 
satellite (1995-present).

\textbf{SAS-2}, \textbf{Second Small Astronomy Satellite}, the first
dedicated NASA \Gray instrument (1972--1973). Mission ended early due
to failure of power supply. First observation of the radio-quiet
Geminga pulsar.

\textbf{SAX or Beppo-SAX}, \textit{Satellite per Astronomia X}, an Italian 
x-ray satellite (1996--2002).

\textbf{SED} --- \textit{Spectral Energy Distribution}, the power an 
instrument would receive as a function of frequency, given the
assumption that its bandwidth is proportional to the frequency.

\textbf{SNR} --- \textit{Super Nova Remnant}, hot material thrown off
as blast wave in supernova explosion. Shocks formed in interaction
with ISM may give rise to particle acceleration, possibly resulting in
a population of charged particles with energies up to $10^{15}$\,eV.

\textbf{TeV} --- \textit{Terra Electron-Volts}, unit of energy equivalent
to $\sim1.6\times10^{-7}$\,J and $1.6$\,erg.

\textbf{UHURU} --- Early NASA x-ray satellite, also known as SAS-1,
(1970--1973).

\textbf{VHE} --- \textit{Very High Energy}, in the context of this work,
the energy range of 300\,GeV to 30\,TeV, accessible to ground-based
\Gray instruments.

\textbf{VLA} --- \textit{Very Large Array}, interferometer consisting of 
27 radio telescopes, each with 25\,m diameter, near Socorro, NM. The
array has four configurations, the largest of which spans an area of
diameter 35\,km.

\textbf{WR-type} --- \textit{Wolf-Rayet}, a star system in with a massive 
O-type star and companion, in which the companion has stripped the star's
outer layers. Spectrum shows high metallicity.

\textbf{XMM-Newton} --- \textit{X-ray Multi-Mirror} mission, a 
high resolution, x-ray instrument operated by the European Space
Agency (1999-present).

\textbf{XRB} --- \textit{X-ray Binary}, a binary system consisting of
a pulsar and a large companion star. Often they are sub-classified as
high-mass (HMXB) or low-mass (LMXB).
